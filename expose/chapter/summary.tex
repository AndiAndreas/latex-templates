\textbf{Hinweise zum Titel der Abschlussarbeit:}
Der Titel der Abschlussarbeit ist deren „Aushängeschild“ und daher sehr wichtig. Er soll prägnant und verständlich formuliert sein. Folgende Fragen sollten helfen dies zu erreichen:
\begin{itemize}
\item Trifft der Titel den geplanten Inhalt der Arbeit? Ist der Titel kurz (maximal Zweizeiler), prägnant und allgemein verständlich?
\item Wird nur eine Sprache verwendet (deutsch oder englisch und nicht denglisch)? Sind die verwendeten Abkürzungen allgemein bekannt oder können Sie nicht auch vermieden werden?
\item Würden Sie sich selbst für die Arbeit nur aufgrund des gewählten Titels interessieren und diese lesen wollen?
\item Würde ihr zukünftiger Arbeitgeber den Titel verstehen?
\end{itemize}
\textbf{Hinweise zum Inhalt des Exposè:}

Das Exposé stellt die Grundlage für ein Arbeitsvorhaben dar, ist die Voraussetzung für die
Anmeldung zur Abschlussarbeit. Das Exposé dient sowohl der eigenen Orientierung als auch der Verständigung zwischen
Kandidatin / Kandidaten und Prüferin/ Prüfer. Auf ein bis zwei Seiten sollten zuerst folgende Punkte erläutert werden (Nicht alle Punkte
sind bei jedem Thema relevant). Die nicht selbstständig getroffenen Aussagen sind mit Literaturquellen zu belegen.\cite{Mangold2003}
\section{Problemstellung}
\begin{itemize}
\item Welches wissenschaftlich oder fachlich relevante Problem ist der Ausgangspunkt der Arbeit und warum handelt es sich dabei um ein Problem?
\item Welche Relevanz hat das untersuchte Problem?
\item Warum ist es lohnenswert diesem Problem nachzugehen? Warum soll ausgerechnet dieses Problem behandelt werden?
\item Wie kommt es zu dem Thema?
\end{itemize}

\section{Fragestellung}
\begin{itemize}
\item Was genau werden Sie selbst untersuchen?
\item Mit diesem Schritt soll das Thema weiter eingegrenzt werden.
\item Auf welche zentrale Frage soll in der Arbeit eine Antwort gefunden oder gegeben werden?
\item Welches konkrete Problem soll damit (aus welcher Perspektive und unter welchen Vorzeichen) behandelt werden? 
\item Hier sollte eine Problemanalyse durchgeführt werden und Teilprobleme identifiziert werden.
\end{itemize}

\section{Ziele/ Hypothesen}
\begin{itemize}
\item Was soll mit den Ausführungen erreicht werden?
\item Was soll belegt oder widerlegt werden?
\item Beide Aspekte müssen mit den vorher aufgestellten (Leit-)Fragen übereinstimmen.
\end{itemize}

\section{Theoriebezug / Forschungsstand}
\begin{itemize}
\item Welche wissenschaftlichen Erkenntnisse liegen zu dem Thema bereits vor?
\item Welche Aspekte des Themas sind bisher noch nicht ausreichend oder erfolgreich behandelt worden? 
\item Auf welche Begriffe, Theorien, Modelle oder Erklärungsansätze soll Bezug genommen werden?
\end{itemize}

\section{Methode}
\begin{itemize}
\item Mit welchen wissenschaftlichen Methoden soll das Problem bzw. Teilprobleme bearbeitet werden?
\item Welche Methoden bieten sich an, die (Leit-)Fragen und Hypothesen angemessen zu
bearbeiten? (theoretisch oder empirisch, qualitativ oder quantitativ, eine Kombination der
Methoden, etc.).
\end{itemize}

\section{Evaluierungsstrategie}
\begin{itemize}
\item Wie sollen die entwickelten Methoden evaluiert werden, so dass nachgewiesen werden
kann, dass das / die Ziel(e) auch erreicht wurde(n).
\end{itemize}






